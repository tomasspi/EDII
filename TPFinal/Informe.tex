\documentclass[12pt,a4paper]{article}
\usepackage[utf8]{inputenc}
\usepackage[spanish]{babel}
\usepackage[T1]{fontenc}
\usepackage{tocbibind} % Bibliografía en el indice
\usepackage{titlesec} % Posibilidad de editar los formatos de chapter
\usepackage{amsmath,amssymb,mathrsfs} % Matemáticas varias
\usepackage{siunitx} % SI de medidas
\usepackage{tabularx} % Para las tablas
\usepackage[title]{appendix} % Para anexos
\usepackage[spanish]{babel} % Para modificar labels por defecto
\renewcommand{\baselinestretch}{1} % Interlineado. 1 es estandar
% --- Arreglos varios para la inclusion de imagenes
\usepackage{float}
\usepackage[pdftex]{graphicx}
\usepackage{subfigure}
\usepackage{graphicx}
\graphicspath{{T:/Tom/Facultad/Logos/}{T:/Tom/Facultad/Materias/EDII/TPs/TPFinal/Diagramas/}}
\usepackage[usenames,dvipsnames]{color}
\DeclareGraphicsExtensions{.png,.jpg,.pdf,.mps,.gif,.bmp}
% --- Para las dimensiones de los márgenes etc
\frenchspacing \addtolength{\hoffset}{-1.5cm}
\addtolength{\textwidth}{3cm} \addtolength{\voffset}{-2.5cm}
\addtolength{\textheight}{4cm}
% --- Para el encabezado
\setlength{\headheight}{33pt}
\usepackage{fancyhdr}
\fancyhead[R]{\includegraphics[height=1cm]{logo_fcefyn_nuevo.jpg}}\fancyhead[L]{\includegraphics[height=1cm]{unc1_a.jpg}}\fancyhead[C]{Electrónica Digital II} \fancyfoot[C]{Página \thepage} \renewcommand{\footrulewidth}{0.4pt}
\pagestyle{fancy}
% --- Para las tablas
\usepackage{multirow} % Juntar filas
\newcolumntype{L}[1]{>{\raggedright\arraybackslash}p{#1}} % Justificar Izq
\newcolumntype{C}[1]{>{\centering\arraybackslash}p{#1}} % Justificar Centrar
\newcolumntype{R}[1]{>{\raggedleft\arraybackslash}p{#1}} % Justificar Der
\usepackage[numbered]{bookmark} % Para que figure las secciones en el PDF
\usepackage{listings} % Para poner código 
\usepackage{enumitem} % Para editar las propiedades de los items
\usepackage{color}
\usepackage[bottom]{footmisc} % Para las notas al pie de la página
\lstset{frame=single} % Código en un cuadro
% --- Para Anexo
\addto\captionsspanish{%
  \renewcommand\appendixname{ANEXO}
  \renewcommand\appendixpagename{ANEXOS}
}
% -------------------------------------------------------- %
% Definicion de colores para el codigo
\lstdefinelanguage{XML}
{
  basicstyle=\ttfamily\footnotesize,
  morestring=[b]",
  moredelim=[s][\bfseries\color{Maroon}]{<}{\ },
  moredelim=[s][\bfseries\color{Maroon}]{</}{>},
  moredelim=[l][\bfseries\color{Maroon}]{/>},
  moredelim=[l][\bfseries\color{Maroon}]{>},
  morecomment=[s]{<?}{?>},
  morecomment=[s]{<!--}{-->},
  commentstyle=\color{DarkOliveGreen},
  stringstyle=\color{blue},
  identifierstyle=\color{red}
}

\renewcommand{\lstlistingname}{Código}

% -------------------------------------------------------- %

\begin{document}

\begin{titlepage}
    \begin{center}
        \vspace*{1cm}
        
        \Large
        \textbf{Universidad Nacional de Córdoba\\
        		Facultad de Ciencias Exatas, Físicas y Naturales}
        
        \vspace{0.5cm}
        \includegraphics[width=0.5\textwidth]{logo_caratula.png}
        
        \vspace{1.5cm}
        
        \textbf{Trabajo Práctico Final}\\
        Electrónica Digital II\\
        Docente: Ing. Martín Del Barco
        
        \vfill  
        
        \vspace{0.8cm}
        

        
        \Large
        Losano Quintana, Juan Cruz\\
        Piñero, Tomás Santiago\\
        Ingeniería en Computación\\
        Año 2019\\
        
        
    \end{center}
\end{titlepage}
% -------------------------------------------------------- %

% --- Tabla de contenidos

\setcounter{secnumdepth}{1}
\setcounter{tocdepth}{4}
\tableofcontents

% -------------------------------------------------------- %

\newpage
\renewcommand{\baselinestretch}{1}
\setlength{\parskip}{0.5em}

\section{Introducción}
	Este informe trata sobre el trabajo práctico final de la materia Electrónica Digital II. El tema es a elección de los estudiantes, por lo que se eligió realizar un sensor de temperatura para propósitos generales.
	
\subsection{Enunciado}
	El sistema a realizar consiste en sensar la temperatura ambiente. La medición comienza cuando se presiona un botón. Una vez presionado se muestra el valor sensado en 4 (cuatro) \emph{displays} de siete segmentos con el formato ``XX$^{\circ}$C'' y se realiza un \emph{log} en la computadora cada cinco segundos.
	
	Para su realización se utilizaron los siguientes materiales:
	
	\begin{itemize}[leftmargin=1.5cm,nosep]
	\item Microcontrolador PIC16F887,
	\item Cristal de \SI{4}{\MHz},
	\item \emph{Bridge USB to UART} CP2102,
	\item Sensor de temperatura LM35,
	\item Cuatro \emph{displays} de siete segmentos cátodo común,
	\item Resistencias de \SI{1}{\kilo\ohm}, \SI{330}{\ohm} y \SI{220}{\ohm},
	\item Capacitor de \SI{22}{\pico\F},
	\item Un pulsador.
	\end{itemize}

\section{Desarrollo}
		
\subsection{Cálculos realizados}

\subsubsection{Resistencias para los segmentos}
	Los \emph{displays} son rojos, usan X voltios y del pic salen X mA, entonces la formula para la resistencia es:
	
	\begin{align*}
	algo &= 0\\
	R &= \frac{algo}{algomas}
	\end{align*}
	
	\begin{equation}
	R = tantos Ohmios
	\end{equation}		
	
\subsubsection{Resistencias para la multiplexación}	
	Al tratarse de \emph{displays} de cátodo común, se necesitan transistores NPN para su multiplexación.
	La fórmula para el tiempo de encendido de cada \emph{display} es la siguiente:
	
	\begin{equation}
	Tiempo = \frac{50ms}{N}
	\end{equation}
	
	 Siendo \emph{N} la cantidad de \emph{displays} a utilizar.
	 
\section{Implementación}
	El PIC tiene como frecuencia de reloj un cristal de \SI{4}{\MHz}, por lo que el ciclo de instrucción se realiza con una frecuencia de \SI{1}{\MHz} debido a que cada 4 ciclos del reloj se realiza una instrucción. Esto se debe a que para ejecutar la instrucción indicada, el PIC debe ejecutar cuatro acciones: 
	
	\begin{enumerate}[leftmargin=1.5cm,nosep]
	\item Buscar la instrucción en la memoria principal.
	\item Decodificar la instrucción.
	\item Ejecutar la instrucción.
	\item Almacenar los resultados.
	\end{enumerate}
	
	Esto es importante para el cálculo de la subrutina de retardo, ya que depende de la frecuencia de reloj que se utilice.

	
\subsection{Diagramas de flujo}
	En esta sección se muestran los diagramas de flujo del programa principal y las subrutinas que utiliza.
	
	\subsubsection{Programa principal}	
	Primero se realiza la configuración de los puertos A y B como salida y entrada digitales, respectivamente. 
	
	Una vez configurados los puertos se toman los datos del puerto de entrada y a esa lectura se la invierte, ya que cuando los Dip Switch estén bajos las entradas estarán con un valor de uno lógico debido a la presencia de las resistencias pull-up.
	
	Consecuentemente el programa almacena los últimos cuatro bits en la variable \emph{numero1} y se lo suma a los primeros cuatro bits leídos, mostrando el resultado en los LEDs de salida. Si el resultado es de cinco bits, el LED del \emph{digit carry} parpadeará y no se podrá realizar otra suma hasta que se resetee el PIC.
	
%	\begin{figure}[H]
%	\includegraphics[scale=0.5]{programa.png}
%	\centering
%	\caption{Diagrama de flujo del programa.}
%	\end{figure}		
	
	\subsubsection{Interrupciones}

	\paragraph{Interrupción por RB0} Esta interrupcion hace esto
	
	\begin{figure}[H]
	\includegraphics[scale=0.5]{RB0.png}
	\centering
	\caption{Diagrama de flujo de la interrupción por RB0.}
	\end{figure}
	
	\paragraph{Interrupción por \emph{Timer0}} Esta interrupcion hace esto
	
	\begin{figure}[H]
	\includegraphics[scale=0.5]{TMR0.png}
	\centering
	\caption{Diagrama de flujo de la interrupción por \emph{Timer0}.}
	\end{figure}
	
	
	\paragraph{Interrupción por \emph{Timer1}} Esta interrupcion hace esto
	
	\begin{figure}[H]
	\includegraphics[scale=0.5]{TMR1.png}
	\centering
	\caption{Diagrama de flujo de la interrupción por \emph{Timer1}.}
	\end{figure}
	
	\paragraph{Interrupción por \emph{ADC}} Esta interrupcion hace esto
	
	\begin{figure}[H]
	\includegraphics[scale=0.5]{ADC.png}
	\centering
	\caption{Diagrama de flujo de la interrupción por \emph{ADC}.}
	\end{figure}
	

\section{Esquema del circuito}

%\includegraphics[angle=90,origin=c,scale=0.33]{circuito.png}
	
\end{document}